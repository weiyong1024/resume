%-------------------------
% Resume in Latex
% Author : Yong Wei
% Template by : Sourabh Bajaj
% License : MIT
%------------------------

\documentclass[letterpaper,11pt]{article}

\usepackage{latexsym}
\usepackage[empty]{fullpage}
\usepackage{titlesec}
\usepackage{marvosym}
\usepackage[usenames,dvipsnames]{color}
\usepackage{verbatim}
\usepackage{enumitem}
\usepackage[hidelinks]{hyperref}
\usepackage{fancyhdr}
\usepackage[english]{babel}
\usepackage{tabularx}
\usepackage[UTF8]{ctex}
\usepackage{setspace}

\pagestyle{fancy}
\fancyhf{} % clear all header and footer fields
\fancyfoot{}
\renewcommand{\headrulewidth}{0pt}
\renewcommand{\footrulewidth}{0pt}

% Adjust margins
\addtolength{\oddsidemargin}{-0.5in}
\addtolength{\evensidemargin}{-0.5in}
\addtolength{\textwidth}{1in}
\addtolength{\topmargin}{-.5in}
\addtolength{\textheight}{1.0in}

\urlstyle{same}

\raggedbottom
\raggedright
\setlength{\tabcolsep}{0in}

% Sections formatting
\titleformat{\section}{
	\vspace{-4pt}\scshape\raggedright\large
}{}{0em}{}[\color{black}\titlerule \vspace{-5pt}]

%-------------------------
% Custom commands
\newcommand{\resumeItem}[2]{
	\item\small{
		\textbf{#1}{: #2 \vspace{-2pt}}
	}
}

\newcommand{\resumeSubheading}[4]{
	\vspace{-1pt}\item
	\begin{tabular*}{0.97\textwidth}[t]{l@{\extracolsep{\fill}}r}
		\textbf{#1} & #2 \\
		\textit{\small#3} & \textit{\small #4} \\
	\end{tabular*}\vspace{-5pt}
}

\newcommand{\resumeSubSubheading}[2]{
	\begin{tabular*}{0.97\textwidth}{l@{\extracolsep{\fill}}r}
		\textit{\small#1} & \textit{\small #2} \\
	\end{tabular*}\vspace{-5pt}
}

\newcommand{\resumeSubItem}[2]{\resumeItem{#1}{#2}\vspace{-4pt}}

\renewcommand{\labelitemii}{$\circ$}

\newcommand{\resumeSubHeadingListStart}{\begin{itemize}[leftmargin=*]}
	\newcommand{\resumeSubHeadingListEnd}{\end{itemize}}
\newcommand{\resumeItemListStart}{\begin{itemize}}
	\newcommand{\resumeItemListEnd}{\end{itemize}\vspace{-5pt}}

%-------------------------------------------
%%%%%%  CV STARTS HERE  %%%%%%%%%%%%%%%%%%%%%%%%%%%%


\begin{document}
	
	%---------SPACING FOR ENGLISH PAGE---------
	\begin{spacing}{0.87}
	
	%----------HEADING-----------------
	\begin{tabular*}{\textwidth}{l@{\extracolsep{\fill}}r}
		\textbf{\href{https://weiyong.org/}{\Large Yong Wei}} & Email : \href{mailto:weiyong1024@gmail.com}{weiyong1024@gmail.com}\\
		\href{https://github.com/weiyong1024}{\textit{Github}} & Mobile : (+86) 186-1263-2313 \\
	\end{tabular*}
	
	
	%-----------EDUCATION-----------------
	\section{Education}
	\resumeSubHeadingListStart
	\resumeSubheading
	{Beihang University}{Beijing, China}
	{Master of Engineering in Electronic and Communications}{Sep. 2016 -- Jan. 2019}
	\resumeSubheading
	{Beihang University (Honors College)}{Beijing, China}
	{Bachelor of Engineering in Electrical and Electronics;  GPA: 3.7}{Sept. 2012 -- Jun. 2016}
	\resumeSubHeadingListEnd
	
	
	%-----------EXPERIENCE-----------------
	\section{Experience}
	\resumeSubHeadingListStart
	
	\resumeSubheading
	{Pony.ai}{Beijing, China}
	{Software Engineer - Infrastructure}{Feb 2019 - Mar 2020}
	\resumeItemListStart
	\resumeItem{Voice Logging Pipeline}
	{The original issue recording pipeline needed an onboard engineer to record the log message via keyboard. So it is necessary to provide a voice logging pipeline to enable single-onboard operator. In the new solution, a pair of microphone(for recording) and hand-button(for trigger) is equipped for each self-driving vehicle (SDV). First imported the opensource library py\_evdev with BSD license to detect input events from hand-button. Then wrote a daemon process based on py\_evdev and ALSA driver from Linux to check the hardware, which monitoring the signal from the hand-button and FIFO from the onboard main process, also triggering/stopping the microphone, then saving the audio files to the disk. For the data processing part, first built the speech-to-text service on Google Cloud Platform, then extracted the content from audio files into text. Finally serialized the meta-info by Google Protobuf and created issues with current issue-reporting pipeline.}
	\resumeItem{Car Sound Workflow}
	{The original car sound module used Pico library by Google to convert the hardcoded text content into audio-stream and then playing it out, which is a waste of onboard computing resources and is limited to English only. The new car sound module was built on a file-based pipeline, aiming to break those limitations. First, built an internal Python command-line tool (CLI) for engineers to add/update and manage sound suite version: For this part, wrote CloudFormation template on AWS to build the Text-to-Speech service, and add the related option for the CLI. Then, add uploading/downloading interface to the storage server and enabling local cache. For the onboard part, added sound names into the sound-request schema, loaded the audio stream into memory when initializing the onboard system, then it would play the sound when receiving the request.}
	\resumeItemListEnd

	
	%--------Multiple Positions Heading------------
%	    \resumeSubSubheading
%	     {Software Engineer I}{Oct 2014 - Sep 2016}
%	     \resumeItemListStart
%	        \resumeItem{Apache Beam}
%	          {Apache Beam is a unified model for defining both batch and streaming data-parallel processing pipelines}
%	     \resumeItemListEnd
%	    \resumeSubHeadingListEnd
	
	
	\resumeSubheading
	{Airbnb}{Beijing, China}
	{Software Engineer Intern - Web Full Stack}{Jun 2018 - Sep 2018}
	\resumeItemListStart
	\resumeItem{Host Retrospect Page}
	{Retrospect page for Airbnb host users. Added related endpoints and mobile web pages on Ruby On Rails framework. Discussed with the designers on the contents and formats of web pages.}
	\resumeItemListEnd
	
	\resumeSubheading
	{Megvii (Face++)}{Beijing, China}
	{Research Intern}{Jan 2018 - Jun 2018}
	\resumeItemListStart
	\resumeItem{Model Search}
	{During the first half, after the dimension of the input graph increased by $14\%$, we aimed to reduce the FLOPs (float operations per second) of the CNN (convolutional neural network) of face recognition module on mobile devices to the same level as before, without damaging the performance of the whole model which is measured by $\frac{1}{10,000}$ passing rate (ROC value when $x=\frac{1}{10,000}$). The problem was solved by adding a bottleneck layer before the Inception-ResNet module to compress channels of the input feature graph by $\frac{1}{2}$. As for the second half, we aimed to improve the performance of our face recognition module referring to state-of-the-art CNN architectures. After experiments on Xception, DenseNet and several other architectures, I proposed a modified version of Google Inception V4 for our production: Instead of using the same atom modules everywhere, different Inception-ResNet modules were used for the shallow, medium and deep part of the CNN. On the other hand, the $N \times N$ kernels of the medium and deep part were replaced by a sequence of $1 \times N$ and $N \times 1$ kernels to improve model capacity. As a result, the $\frac{1}{10,000}$ passing rate was improved by $1\%$ on most benchmarks.}
	\resumeItemListEnd
	
	\resumeSubheading
	{Beihang University}{Beijing China}
	{Research}{Jul 2015 - Mar 2016}
	\resumeItemListStart
	\resumeItem{The $5^{th}$ Generation Mobile Network}
	{Proposed an channel estimation method based on uplink wireless data and channel sparcity with supervisor, improving upper bound of the wireless system's throughput by $28\%$ according to simulation results.\\Paper published in Journal of Signal Processing (First author). DOI: 10.16798/j.issn.1003-0530.2017.06.002}
	\resumeItemListEnd
	
	\resumeSubHeadingListEnd

	
	%-----------Competitions-----------------
	\section{Competitive programming}
	\resumeSubHeadingListStart
	\resumeSubheading
	{Google Code Jam Kickstart 2017 Round F}{}
	{Top $5\%$, rank $108^{th}$ globally. \href{https://codejam.withgoogle.com/codejam/contest/7254486/scoreboard\#sp=91}{\textbf{scoreboard, id:WeiYong1024}}}{}

	\resumeSubHeadingListEnd
	
	
	%-----------PROJECTS-----------------
%	\section{Projects}
%	\resumeSubHeadingListStart
%	\resumeSubItem{QuantSoftware Toolkit}
%	{Open source python library for financial data analysis and machine learning for finance.}
%	\resumeSubItem{Github Visualization}
%	{Data Visualization of Git Log data using D3 to analyze project trends over time.}
%	\resumeSubItem{Recommendation System}
%	{Music and Movie recommender systems using collaborative filtering on public datasets.}
%	\resumeSubItem{Mac Setup}
%	{Book that gives step by step instructions on setting up developer environment on Mac OS.}
%	\resumeSubHeadingListEnd
	
	
	%--------PROGRAMMING SKILLS------------
	\section{Programming Skills}
	  \resumeSubHeadingListStart
	    \item{
	      \textbf{Languages}{: Use C++, Python most commonly; Have experience in Shell, JavaScript, }
	      \hfill
	    }
    	\item{
    	  \textbf{Technologies}{: Have experience in AWS, Google Cloud, Kubernetes, React}
    	  \hfill
    	}
	  \resumeSubHeadingListEnd
	  	
	\end{spacing}
	
	
	%-------------------------------------------
	
	\newpage
	
	%---------SPACING FOR CHINESE PAGE---------
	\begin{spacing}{0.96}
		
		%----------HEADING-----------------
		\begin{tabular*}{\textwidth}{l@{\extracolsep{\fill}}r}
			\textbf{\href{https://weiyong.org/}{\Large 魏雍}} & 邮箱 : \href{mailto:weiyong1024@gmail.com}{weiyong1024@gmail.com}\\
			\href{https://github.com/weiyong1024}{\textit{Github}} & Mobile : (+86) 186-1263-2313 \\
		\end{tabular*}
		
		
		%-----------EDUCATION-----------------
		\section{教育背景}
		\resumeSubHeadingListStart
		\resumeSubheading
		{北京航空航天大学}{中国,北京}
		{电子与通信工程硕士}{2016年1月 -- 2019年1月}
		\resumeSubheading
		{北京航空航天大学 (沈元荣誉学院)}{中国,北京}
		{电子信息工程硕士; GPA: 3.7}{2012年9月 -- 2016年6月}
		\resumeSubHeadingListEnd
		
		
		%-----------EXPERIENCE-----------------
		\section{工作经历}
		\resumeSubHeadingListStart
		
		\resumeSubheading
		{小马智行}{中国,北京}
		{软件工程师 - 基础架构}{2019年2月 - 2020年3月}
		\resumeItemListStart
		\resumeItem{行车录音工具链}
		{原有issue记录流程需要一位工程师跟车通过键盘记录问题的描述信息,故需要增加通过语音记录issue的流程供司机使用以支持单人单车运营。为此新增麦克风和按钮作为硬件方案,首先在Bazel项目中引入基于BSD软件许可的输入设备接口库evdev,并基于该库和Linux的ALSA音频驱动实现负责硬件检测、监听来自外部按钮的信号和车载系统进程的管道信息、触发与停止麦克风录音的守护进程sound\_recorder,在行车过程中将issue信息以音频文件的形式存储下来。在数据处理阶段,首先在Google Cloud Platform上搭建语音转文字服务,在本地使用Python脚本将音频文件的内容提取出来,然后使用Google的Protobuf工具将原信息序列化。最后使用已有的Issue汇报系统流程将相关问题发送给QA同事。}
		\resumeItem{车载语音系统}
		{原有的车载语音模块使用 Google 的 Pico 文字转语音工具库在行车过程中将硬编码的语音内容文本实时播放,从而导致文字转语音的过程需要耗费车载计算资源,同时语言被限制只能使用英文。新的车载语音系统使用基于音频文件的工作流程,旨在减少计算量并支持语音的I18N。为此首先用Python编写内部click命令行工具car\_sound\_utils(以下简称CLI),供工程师用于添加和升级现有语料库、管理语音包版本:首先,在AWS上编写CloudFormation模板搭建语音转文字的Web服务,并在CLI中添加对应的调用命令和相应参数,用以在本地生成多语言语音语料。然后,在CLI中添加上传、下载语音文件和上传语音包到内部storage服务器的接口,并加入本地缓存机制加速下载。对车载系统部分,在播放语音请求的消息原型中加入语音名称,初始化语音模块时使用Linux的ALSA驱动将音频流加载进内存,并在语音模块接到播放请求消息时播放。}
		\resumeItemListEnd
		
		
		%--------Multiple Positions Heading------------
		%	    \resumeSubSubheading
		%	     {Software Engineer I}{Oct 2014 - Sep 2016}
		%	     \resumeItemListStart
		%	        \resumeItem{Apache Beam}
		%	          {Apache Beam is a unified model for defining both batch and streaming data-parallel processing pipelines}
		%	     \resumeItemListEnd
		%	    \resumeSubHeadingListEnd
		
		
		\resumeSubheading
		{爱彼迎}{中国,北京}
		{软件工程师实习生 - Web全栈}{2018年6月 - 2018年9月}
		\resumeItemListStart
		\resumeItem{房东回顾页面}
		{开发房东回顾页面。在Ruby On Rails框架下实现相关的Endpoints和移动段前段页面,与设计师讨论确定前端页面的样式和内容。}
		\resumeItemListEnd
		
		\resumeSubheading
		{旷视科技}{中国,北京}
		{炼丹实习生}{2018年1月 - 2018年6月}
		\resumeItemListStart
		\resumeItem{模型搜索}
		{在实习期前半段,由于用输入网络的图片空间维度增加了$14\%$,我们希望能够在不牺牲模型性能的前提下,通过模型压缩将的CNN(卷积神经网络)的FLOPs(每秒钟浮点运算量)降低到先前的水平。这里模型性能使用万一通过率(ROC 曲线上$\frac{1}{10,000}$时纵坐标的值)来衡量。最终通过对网络中 Inception-ResNet 模块加入 bottleneck 层将输入特征图层数压缩一半的方式使该问题得以解决。在实习期后半段,我们希望借鉴最新的 CNN 结构研究成果进一步提升用于移动端人脸识别模块的性能。在对一些经典网络结构如 Xception、DenseNet 进行测试以后,我提出了一种基于 Google Inception V4 的网络结构用于产品:首先,先前网络中全部使用相同的 Inception-ResNet 模块,而我针对网络中浅层、中 层、深层使用不同结构的模块。另一方面,我通过将$N \times N$卷积核替换为$1 \times N$与$N \times 1$卷积核串联的方式增加了网络容量。最终实验结果显示,上述网络结构使得模型性能在大多数数据集上提升了一个百分点。}
		\resumeItemListEnd
		
		\resumeSubheading
		{北京航空航天大学}{中国,北京}
		{硕士研究生}{2015年1月 - 2016年5月}
		\resumeItemListStart
		\resumeItem{第五代移动通信网络}
		{在导师指导下提出一种利用上行数据及信道稀疏特性提升信道估计质量的方法,使得网络吞吐量提升了$28\%$。\\以第一作者合作发表于中文核心期刊《信号处理》。DOI: 10.16798/j.issn.1003-0530.2017.06.002}
		\resumeItemListEnd
		
		\resumeSubHeadingListEnd
		
		
		%-----------Competitions-----------------
		\section{Coding能力}
		\resumeSubHeadingListStart
		\resumeSubheading
		{Google Code Jam Kickstart 2017 Round F}{}
		{前 $5\%$, 全球排名 $108^{th}$。 \href{https://codejam.withgoogle.com/codejam/contest/7254486/scoreboard\#sp=91}{\textbf{计分板链接(id:WeiYong1024)}}}{}
		
		\resumeSubHeadingListEnd
		
		
		%-----------PROJECTS-----------------
		%	\section{Projects}
		%	\resumeSubHeadingListStart
		%	\resumeSubItem{QuantSoftware Toolkit}
		%	{Open source python library for financial data analysis and machine learning for finance.}
		%	\resumeSubItem{Github Visualization}
		%	{Data Visualization of Git Log data using D3 to analyze project trends over time.}
		%	\resumeSubItem{Recommendation System}
		%	{Music and Movie recommender systems using collaborative filtering on public datasets.}
		%	\resumeSubItem{Mac Setup}
		%	{Book that gives step by step instructions on setting up developer environment on Mac OS.}
		%	\resumeSubHeadingListEnd
		
		
		%--------PROGRAMMING SKILLS------------
		\section{技术栈}
		\resumeSubHeadingListStart
		\item{
			\textbf{编程语言}{: C++, Python最常用;使用过Shell, JavaScript}
			\hfill
		}
		\item{
			\textbf{工具}{: 使用过AWS, Google Cloud, Kubernetes, React}
			\hfill
		}
		\resumeSubHeadingListEnd
		
	\end{spacing}



\end{document}